% Options for packages loaded elsewhere
\PassOptionsToPackage{unicode}{hyperref}
\PassOptionsToPackage{hyphens}{url}
%
\documentclass[
]{article}
\usepackage{amsmath,amssymb}
\usepackage{lmodern}
\usepackage{iftex}
\ifPDFTeX
  \usepackage[T1]{fontenc}
  \usepackage[utf8]{inputenc}
  \usepackage{textcomp} % provide euro and other symbols
\else % if luatex or xetex
  \usepackage{unicode-math}
  \defaultfontfeatures{Scale=MatchLowercase}
  \defaultfontfeatures[\rmfamily]{Ligatures=TeX,Scale=1}
\fi
% Use upquote if available, for straight quotes in verbatim environments
\IfFileExists{upquote.sty}{\usepackage{upquote}}{}
\IfFileExists{microtype.sty}{% use microtype if available
  \usepackage[]{microtype}
  \UseMicrotypeSet[protrusion]{basicmath} % disable protrusion for tt fonts
}{}
\makeatletter
\@ifundefined{KOMAClassName}{% if non-KOMA class
  \IfFileExists{parskip.sty}{%
    \usepackage{parskip}
  }{% else
    \setlength{\parindent}{0pt}
    \setlength{\parskip}{6pt plus 2pt minus 1pt}}
}{% if KOMA class
  \KOMAoptions{parskip=half}}
\makeatother
\usepackage{xcolor}
\usepackage[margin=1in]{geometry}
\usepackage{color}
\usepackage{fancyvrb}
\newcommand{\VerbBar}{|}
\newcommand{\VERB}{\Verb[commandchars=\\\{\}]}
\DefineVerbatimEnvironment{Highlighting}{Verbatim}{commandchars=\\\{\}}
% Add ',fontsize=\small' for more characters per line
\usepackage{framed}
\definecolor{shadecolor}{RGB}{248,248,248}
\newenvironment{Shaded}{\begin{snugshade}}{\end{snugshade}}
\newcommand{\AlertTok}[1]{\textcolor[rgb]{0.94,0.16,0.16}{#1}}
\newcommand{\AnnotationTok}[1]{\textcolor[rgb]{0.56,0.35,0.01}{\textbf{\textit{#1}}}}
\newcommand{\AttributeTok}[1]{\textcolor[rgb]{0.77,0.63,0.00}{#1}}
\newcommand{\BaseNTok}[1]{\textcolor[rgb]{0.00,0.00,0.81}{#1}}
\newcommand{\BuiltInTok}[1]{#1}
\newcommand{\CharTok}[1]{\textcolor[rgb]{0.31,0.60,0.02}{#1}}
\newcommand{\CommentTok}[1]{\textcolor[rgb]{0.56,0.35,0.01}{\textit{#1}}}
\newcommand{\CommentVarTok}[1]{\textcolor[rgb]{0.56,0.35,0.01}{\textbf{\textit{#1}}}}
\newcommand{\ConstantTok}[1]{\textcolor[rgb]{0.00,0.00,0.00}{#1}}
\newcommand{\ControlFlowTok}[1]{\textcolor[rgb]{0.13,0.29,0.53}{\textbf{#1}}}
\newcommand{\DataTypeTok}[1]{\textcolor[rgb]{0.13,0.29,0.53}{#1}}
\newcommand{\DecValTok}[1]{\textcolor[rgb]{0.00,0.00,0.81}{#1}}
\newcommand{\DocumentationTok}[1]{\textcolor[rgb]{0.56,0.35,0.01}{\textbf{\textit{#1}}}}
\newcommand{\ErrorTok}[1]{\textcolor[rgb]{0.64,0.00,0.00}{\textbf{#1}}}
\newcommand{\ExtensionTok}[1]{#1}
\newcommand{\FloatTok}[1]{\textcolor[rgb]{0.00,0.00,0.81}{#1}}
\newcommand{\FunctionTok}[1]{\textcolor[rgb]{0.00,0.00,0.00}{#1}}
\newcommand{\ImportTok}[1]{#1}
\newcommand{\InformationTok}[1]{\textcolor[rgb]{0.56,0.35,0.01}{\textbf{\textit{#1}}}}
\newcommand{\KeywordTok}[1]{\textcolor[rgb]{0.13,0.29,0.53}{\textbf{#1}}}
\newcommand{\NormalTok}[1]{#1}
\newcommand{\OperatorTok}[1]{\textcolor[rgb]{0.81,0.36,0.00}{\textbf{#1}}}
\newcommand{\OtherTok}[1]{\textcolor[rgb]{0.56,0.35,0.01}{#1}}
\newcommand{\PreprocessorTok}[1]{\textcolor[rgb]{0.56,0.35,0.01}{\textit{#1}}}
\newcommand{\RegionMarkerTok}[1]{#1}
\newcommand{\SpecialCharTok}[1]{\textcolor[rgb]{0.00,0.00,0.00}{#1}}
\newcommand{\SpecialStringTok}[1]{\textcolor[rgb]{0.31,0.60,0.02}{#1}}
\newcommand{\StringTok}[1]{\textcolor[rgb]{0.31,0.60,0.02}{#1}}
\newcommand{\VariableTok}[1]{\textcolor[rgb]{0.00,0.00,0.00}{#1}}
\newcommand{\VerbatimStringTok}[1]{\textcolor[rgb]{0.31,0.60,0.02}{#1}}
\newcommand{\WarningTok}[1]{\textcolor[rgb]{0.56,0.35,0.01}{\textbf{\textit{#1}}}}
\usepackage{graphicx}
\makeatletter
\def\maxwidth{\ifdim\Gin@nat@width>\linewidth\linewidth\else\Gin@nat@width\fi}
\def\maxheight{\ifdim\Gin@nat@height>\textheight\textheight\else\Gin@nat@height\fi}
\makeatother
% Scale images if necessary, so that they will not overflow the page
% margins by default, and it is still possible to overwrite the defaults
% using explicit options in \includegraphics[width, height, ...]{}
\setkeys{Gin}{width=\maxwidth,height=\maxheight,keepaspectratio}
% Set default figure placement to htbp
\makeatletter
\def\fps@figure{htbp}
\makeatother
\setlength{\emergencystretch}{3em} % prevent overfull lines
\providecommand{\tightlist}{%
  \setlength{\itemsep}{0pt}\setlength{\parskip}{0pt}}
\setcounter{secnumdepth}{-\maxdimen} % remove section numbering
\ifLuaTeX
  \usepackage{selnolig}  % disable illegal ligatures
\fi
\IfFileExists{bookmark.sty}{\usepackage{bookmark}}{\usepackage{hyperref}}
\IfFileExists{xurl.sty}{\usepackage{xurl}}{} % add URL line breaks if available
\urlstyle{same} % disable monospaced font for URLs
\hypersetup{
  pdftitle={Modélisation Prédictive Rapport},
  pdfauthor={Valentin Gölz, Laura Fuentes},
  hidelinks,
  pdfcreator={LaTeX via pandoc}}

\title{Modélisation Prédictive Rapport}
\author{Valentin Gölz, Laura Fuentes}
\date{February 14, 2023}

\begin{document}
\maketitle

Le premier reflèxe est ici de télécharger l'ensemble des packages dont
on fera usage lors du développement des différents modèles.\\
\# This is bigger text

On télécharge ensuite les deux jeux de données

\begin{Shaded}
\begin{Highlighting}[]
\FunctionTok{load}\NormalTok{(}\StringTok{"Data/Data0.Rda"}\NormalTok{)}
\FunctionTok{load}\NormalTok{(}\StringTok{"Data/Data1.Rda"}\NormalTok{)}

\NormalTok{sel\_a }\OtherTok{\textless{}{-}} \FunctionTok{which}\NormalTok{(Data0}\SpecialCharTok{$}\NormalTok{Year}\SpecialCharTok{\textless{}=}\DecValTok{2019}\NormalTok{)}
\NormalTok{sel\_b }\OtherTok{\textless{}{-}} \FunctionTok{which}\NormalTok{(Data0}\SpecialCharTok{$}\NormalTok{Year}\SpecialCharTok{\textgreater{}}\DecValTok{2019}\NormalTok{)}
\end{Highlighting}
\end{Shaded}

Le but ici est de construire un modèle qui permettant de prédire la
consommation française en énergie pendant la période du Covid. Pour
comprendre et apprehender le cadre d'étude on commencera par effectuer
un modèle simple. C'est-à-dire un modèle linéaire à deux covariables,
une quantitative et une autre catégorique. La variable WeekDays2 est une
version modifiée de la variable WeekDays qui distingue les jours
laborales, samedis et dimanches.

\begin{Shaded}
\begin{Highlighting}[]
\NormalTok{mod1 }\OtherTok{=} \FunctionTok{lm}\NormalTok{(Load}\SpecialCharTok{\textasciitilde{}}\NormalTok{WeekDays2}\SpecialCharTok{+}\NormalTok{Temp, }\AttributeTok{data=}\NormalTok{Data0[sel\_a,])}
\FunctionTok{summary}\NormalTok{(mod1)}
\end{Highlighting}
\end{Shaded}

\begin{verbatim}
## 
## Call:
## lm(formula = Load ~ WeekDays2 + Temp, data = Data0[sel_a, ])
## 
## Residuals:
##      Min       1Q   Median       3Q      Max 
## -16071.4  -3480.3    -21.9   3256.1  18573.0 
## 
## Coefficients:
##                   Estimate Std. Error  t value Pr(>|t|)    
## (Intercept)       76284.60     294.73  258.825  < 2e-16 ***
## WeekDays2Monday   -1153.50     324.92   -3.550 0.000391 ***
## WeekDays2Saturday -5298.50     326.07  -16.250  < 2e-16 ***
## WeekDays2Sunday   -8014.38     325.88  -24.593  < 2e-16 ***
## WeekDays2WorkDay    450.09     265.74    1.694 0.090423 .  
## Temp              -1572.77      14.36 -109.510  < 2e-16 ***
## ---
## Signif. codes:  0 '***' 0.001 '**' 0.01 '*' 0.05 '.' 0.1 ' ' 1
## 
## Residual standard error: 4708 on 2939 degrees of freedom
## Multiple R-squared:  0.8191, Adjusted R-squared:  0.8188 
## F-statistic:  2661 on 5 and 2939 DF,  p-value: < 2.2e-16
\end{verbatim}

Pour construire l'équation, nous nous commes d'une part concentrés sur
le poids des coefficients au niveau de la régression linéaire. D'autre
part on a considéré les effets des jours de la semaine sur la
consommation du jour précédent.

Dans la suite on utilisera le package qgam, et en particulier la
fonction qgam. Celle-ci est performe une régression quantile. En fixant
le quantile à 0.4, on a changé la fonction de perte. On a ainsi
introduit un biais, qui permet de s'ajuster mieux aux données du covid.

On a ainsi décidé de garder notre équation sur la qgam et de
l'implémenter ensuite sur d'autres modèles. On a ainsi d'étudier les
résidus. Ceci va ainsi nous permettre de ????

Après avoir vu en cours les forêts aléatoires, on a décidé de
l'implémenter sur \ldots\ldots{}

Nous avons enfin appliqué arima sur les résultats obtenus, pour gagner
en pécision.

\begin{Shaded}
\begin{Highlighting}[]
\CommentTok{\#ts\_res\_forecast \textless{}{-} ts(c(Block\_residuals.ts, Data\_test$Load{-}gam9.forecast),  frequency= 7)}
\CommentTok{\#refit \textless{}{-} Arima(ts\_res\_forecast, model=fit.arima.res)}
\CommentTok{\#prevARIMA.res \textless{}{-} tail(refit$fitted, nrow(Data\_test))}
\CommentTok{\#gam9.arima.forecast \textless{}{-} gam9.forecast + prevARIMA.res}
\end{Highlighting}
\end{Shaded}

Comme dernière méthode, nous avons décidé de mettre en place un
aggrégation d'experts pour extraire une combinaison de prédicteurs qui
puissent améliorer davantage la performance du modèle. Pour cela, nous
avons regroupé les différents prédicteurs dans une variable experts, et
nous avons\ldots. Nous avons également décidé d'ajouter un modèle
additionnel utilisant le filtre kalman. Nous avions déjà essayer un tel
élément, mais les résultats n'étaient pas tellement satisfaisants.

\end{document}
